Before you dive in, you ought to get excited about data visualization. While there is a glaring lack of major museum space devoted to data visualization (I just recall a disappointing exhibit at the Cooper Hewitt), you will find many wonderful displays if you only keep your eyes peeled. 
 
If you like to listen to people talk about data visualization, I recommend the \link{https://datastori.es}{Data Stories podcast}. 

If you'd like to start by reading one of the pioneers, check out \link{https://www.edwardtufte.com/tufte/}{Edward Tufte}, who continues to write new material. For more explicit or domain-specific guidance than Tufte might provide, see \link{https://www.storytellingwithdata.com/book/downloads}{Storytelling with Data} by Cole Nussbaumer Knaflic or \link{http://cup.columbia.edu/book/better-data-visualizations/9780231193115}{Better Data Visualization} by Jonathan Schwabish. Many of Schwabish's main themes are also communicated more briefly in \cite{schwabish2014economist}. I have limited patience for how-to guides when they edge toward being overly prescriptive (I've never read any books on how to write well either), but I've profited from these titles. They are useful for their treatment of fundamentals like preattentive processing and surfacing more variety in visualizations, helping to inspire a richer repertoire. Knaflic's book is oriented toward business professionals and Schwabish adds his own public policy background. As a result, Knaflic concentrates on what I call prosaic visuals  and Schwabish pushes further into the realm of poetry. Schwabish discusses the tradeoffs between standard and nonstandard graphs, noting that novelty can encourage more active processing, providing further justification for using a less accurate graph in select, exploratory cases.
%Media outlets 

Media outlets like the New York Times and Wall Street Journal make usually good use of data visualization. Take appropriate inspiration these sources and from the  \link{https://www.reddit.com/r/dataisbeautiful}{r/DataIsBeautiful} and \link{https://www.reddit.com/r/dataisugly/}{r/DataIsUgly} subreddits.

The official matplotlib documentation continues to be a great resource, especially with new tutorials and galleries being added since I began work on this book. There is also a good Data Visualization section in \link{https://aeturrell.github.io/coding-for-economists/intro.html}{\emph{Coding for Economists}} by Arthur Turrell. For a more advanced treatment of matplotlib, check out \link{https://github.com/rougier/scientific-visualization-book}{Scientific Visualization: Python + Matplotlib}.
