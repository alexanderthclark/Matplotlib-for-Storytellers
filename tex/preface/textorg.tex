Continuing the \hyperref[sec:writing]{parallel to writing}, I have built this text around two main parts: \hyperref[part:prose]{Prose} and \hyperref[part:poetry]{Poetry}, though the distinction between prose and poetry is surely less exact than the division I've created. Prose, or Part \ref{part:prose}, focuses on the fundamentals of customizing plots through the object-oriented interface. This section attempts to be reasonably thorough in breadth while providing only a minimal effective dose in depth. Then, after a mathematical interlude in Part \ref{part:math}, we reach poetry in Part \ref{part:poetry}. There can be no comprehensiveness to this section. I provide a guide to drawing in matplotlib, mostly with various \link{https://matplotlib.org/stable/api/artist_api.html\#artist-class}{artist} objects. The mathematical interlude is there for those who would like to review some trigonometry I use. Then, I introduce two special (for fun) topics in Part \ref{part:topics}, multi-dimensional scaling and ternary plots. 
%For those with the time and patience, I do recommend reading through the sections in order. The text isn't structured like a reference book, but you may still treat it as such. 
% is this a reference book? it's more expository

%If you're in a hurry, I recommend reading Chapter \ref{chapter:oop}, then Chapter \ref{chapter:StyleConfig}, and then you can figure out how to skip around or use other resources to fill in the gaps. 