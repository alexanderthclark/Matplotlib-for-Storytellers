% Visualization 
% Prose, Poetry, and the Invasion of Your Mind by Ready-made Phrases
% Good Visualization is like Good Writing

This book isn't a guide to visualization design, but we must consider, at least briefly, what makes for good visualization and then why you might find matplotlib useful in that pursuit. 

Data visualization is a form of communication not much different than writing. Cole Nussbaumer Knaflic's \emph{Storytelling with Data} parallels writing style guides like Sir Ernest Gowers' \emph{The Complete Plain Words}. They both emphasize clarity and stripping out what is not essential. Matplotlib doesn't offer any unique advantage in pursuing clarity. Instead, the advantage is a tactical one. Matplotlib will expand your options. Sometimes straightforward prose is appropriate and sometimes only poetry will be stirring enough to capture your audience's attention. There exist prosaic visualizations and poetic visualizations with all the same tradeoffs. 

Prose is precise and direct. Poetry has a certain beauty that invites interest and mediates higher truths. The familiar bar chart is prose, plainly reporting the numbers that need to be reported. Your boss will appreciate prose in a routine meeting. But imagine the king must wrestle with a difficult truth. Prose won't do. Only a jester or a Shakespearean fool can deliver the message and only by rhyme and riddle. So it may be with your C-level audience. The small truths of your bar charts don't matter to a busy CEO. Easier said than done, but capture your CEO's attention with a poetic visualization that might sacrifice some precision for its larger message. 

%In writing, we develop our capacity for good prose and good poetry by learning the fundamentals and then by practice and reading. It's a larger task than this book tackles, but this book aims to better your matplotlib fundamentals.

A hurdle to crafting good visualizations is being limited to a short menu of cookie cutter graphics, whatever is available in Excel, a dashboard tool, or from a limited knowledge of matplotlib. Ahead of us is the chance to break free from those cookie cutter, ready-made visuals. In writing, George Orwell made good note of the ``invasion of one's mind by ready-made phrases,'' in his worthwhile essay \textit{Politics and the English Language}: 

\begin{displayquote}
\textins{Ready-made phrases} will construct your sentences for you---even think your thoughts for you, to a certain extent---and at need they will perform the important service of partially concealing your meaning even from yourself.
%\hspace{10pt} From \textit{Politics and the English Language} by George Orwell
\end{displayquote}

\noindent The important point here is that the unimaginative application of ready-made visualizations, just like phrases, can conceal your meaning from yourself, not to mention your audience, and create a monotonous presentation of bar chart after bar chart. 

The parallels between writing and making visuals go one level further. If you want to \emph{become} a good writer, you will learn grammar, read good writers who came before you, write a lot, and skirt the rules a bit as you find your voice. In other words, you will do many things. Data visualization is no different. In what follows, you will begin to master just one thing, the technical grammar of matplotlib. %Go forth, and imagine Orwell has taken up data science in the afterlife and his ghost is watching you prepare your visualizations.

%I have no idea what this graph from @F1 (Formula 1) on Twitter is supposed to convey. 
%\begin{center}
%    \includegraphics[width = .55\textwidth]{F1}
%\end{center}
%from philosopher Roger Scruton in mind, ``Art moves us because it is beautiful, and it is beautiful because it means something. It can be meaningful without being beautiful; but to be beautiful it must be meaningful.''